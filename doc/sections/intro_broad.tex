%Biologging technology now provides researchers with kinematic data collected at extremely high frequencies in time \citep{Hooten:2017}.
%The collection and analysis of data from devices such as accelerometers have brought new insights to areas ranging from monitoring machine health \citep{Getman:2009} to understanding physical activity levels in children \citep{Morris:2007}. 
%The study of animal movement in particular has been transformed by tracking devices that record kinematic information in a variety of environments \citep{Borger:2020,Dot:2016b}. Tags can record over 50 observations per second, resulting in time series that contain millions of observations over the course of several hours. These high-frequency data sets contain a wealth of information about fine-scale human and animal behaviour, but developing computationally efficient and accurate models these large data sets poses a challenge for statisticians and biologists.

Recent advances in tracking technology allows ecologists to collect an unprecedented amount of movement and kinematic data for a wide variety of animals \citep{Patterson:2017}. These data sets allow researchers to detect hunting behaviour \citep{Heerah:2017}, understand habitat selection \citep{Michelot:2019b}, and develop activity budgets \citep{Dot:2016} for these animals. Understanding an animal's movement and behaviour can assist in their conservation \citep{Sutherland:1998}.

Animal movement data is collected on a wide variety of time scales and with varying regularity. For example, some animal position data can be sampled irregularly at an average rate of less than one observation per an hour \citep{Gryba:2019}, while some accelerometers collect data regularly at a rate of 50 observations per second \citep{Daneault:2021}. Partially in response to this wide variety and plethora of data, statistical models for animal movement and behaviour have become more complicated in recent years \citep{Hooten:2017}. These models have therefore become more difficult to perform parameter inference over \citep{McClintock:2012,Michelot:2019b}.

One of the most prevalent models used to describe animal movement is the hidden Markov model, or HMM \citep{McClintock:2020}. Although they are effective on relatively coarse scales, standard HMMs rely on several assumptions that are often unrealistic for fine-scale processes. For example, traditional HMMs assume that the process dictating the behaviour of an animal does not change over time. This assumption is violated if some coarse-scale process (e.g. dive type for a marine animal) affects the animal's fine-scale behaviour (e.g. searching for prey vs chasing prey). In addition, HMMs model observations as independent of one another when conditioned on the animal's underlying behaviour. However, fine-scale processes usually exhibit intricate dependence structures with a high degree of autocorrelation (e.g., the acceleration of an animal at a given time is highly correlated with its acceleration 20 milliseconds later). One additional concern with most HMMs used in animal movement modelling is that researchers do not know which behavioural states the HMM will discover \textit{a priori}. Instead, ecologists must interpret an animal's behavioural states after the HMM estimates them.

While high-frequency data sets can reveal fine-scale animal behaviours, many location data sets are sparse and irregularly sampled in time. Marine animal location in particular is often recorded using satellites \citep{Gryba:2019} or by researchers visually observing the animal when it surfaces \citep{Hartman:2020}. Hidden Markov models fail when observations are not equi-spaced in time, so many ecological statisticians instead implement continuous-time methods based on diffusion processes \citep{Blackwell:2016, Michelot:2019b}. Movement models based on diffusion processes can handle irregular time intervals, but they are often seen as more difficult to fit and less interpretable compared to discrete-time models.

Parameter estimation for both HMMs and continuous-time methods that account for complex animal behaviours is computationally expensive. For example, the hierarchical hidden Markov model (HHMM) from the case study of chapter 2 requires trial-and-error to perform model selection and takes hours to fit. In addition, sparsely sampled diffusion processes often have intractable likelihoods. Researchers can approximate the likelihood by inferring an animal's position between observations, but inferring unobserved positions can be computationally expensive if many unobserved positions are inferred between observations \citep{Lindstrom:2012}.

%Relatively recent advances in biologging technology has resulted in an explosion of increasingly high-frequency and large-scale data sets detailing the location and movement of humans as well as a wide variety of animals \cite{Patterson:2017}. These rich data sets give researchers the opportunity to described animal behaviour in extremely fine detail. However, these data sets also pose a challange for statiticians and biologists since increasingly complicated models are required to sufficiently describe the fine-scale processes reflected within. 

The examples above highlight the importance of developing accurate statistical models and efficient inference methods for animal movement at a variety of scales. As such, the overall goals of this thesis are: (1) to implement statistical models that accurately describe animal movement and behaviour using data sets on a variety of scales, and (2) to develop novel inference procedures to mitigate the computational burden of implementing these statistical models.

The primary case study I will use in this thesis involves modelling the movement and behaviour of northern and southern resident killer whales (\textit{Orcinus orca}) off the coast of British Columbia, Canada. The northern resident population is slowly growing and is designated as threatened, while the southern resident population is classified as endangered \citep{DFO:2018}. Fitting accurate statistical models to this data set will help researchers understand each killer whale's energy expenditure as well as discrepancies between the two sub-populations \citep{Green:2009,Dot:2016,Wilson:2019}. Understanding energy expenditure may also aid in conservation efforts for both the northern and southern residents \citep{Noren:2011}.

This thesis proposal is organized into a total of six chapters. 
%
In chapter 2, I review hidden Markov models and several variations recently introduced in the animal movement literature. I describe a novel approach to synthesize these methods to model fine-scale movement processes. I construct several movement models and compare their performance when applied to kinematic data from a northern resident killer whale in 2019. 
%
In chapter 3, I highlight the advantages of using labelled data and introduce a \textit{partially} hidden Markov model, or PHMM, that explicitly models the probability of observing a label at a given time. I consider possible issues when labels are either sparse or preferentially observed, and propose several methods that account for these issues.
%
In chapter 4, I review several inference methods for HMMs and discuss their computational complexity. I then leverage the structure of a PHMM to propose a new inference method that uses labels to reduce the number of computations per parameter update. I generalize this inference method to standard HMMs with no observed labels.
%
In chapter 5, I review the general structure of continuous time methods in statistical ecology as well as common inference methods used to infer their parameters. I then propose a new inference method based on a multi-scale method introduced by \citet{Kou:2012} and non-reversible parallel tempering as described in \citet{Syed:2019}.
%
In chapter 6, I summarize my preliminary results and discuss further work.